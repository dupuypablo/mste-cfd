\documentclass{CFD2011}
\usepackage{CFD2011}


%%%%%%%%%%%%%%%%%%%%%%%%%%%%%%%%%%%%%%%%%%%%%%%%%%%%%%%%%%%%%%%%%%%%%%%%%
%This template is created for complying with the author instructions for the 
%8th International Conference on CFD in Oil & Gas, Metallurgical and Process Industries
%hosted by SINTEF/NTNU, Trondheim Norway
%21-23 June 2011
%
%Sverre G. Johnsen (sverre.g.johnsen@sintef.no)
%SINTEF Materials and Chemistry


%Required files:
%   ExampleFile.tex
%   xampleFile.nls
%   CFD2011.bst
%   CFD2011.cls
%   CFD2011.sty
%   References.bib

%Example-specific files:
%   figure.eps
%   Table.tex
%%%%%%%%%%%%%%%%%%%%%%%%%%%%%%%%%%%%%%%%%%%%%%%%%%%%%%%%%%%%%%%%%%%%%%%%%




\title{Industrial Applications of CFD}
\paperID{CFD11-52}
\author{Ola}{Nordmann} %{forename}{surname}
\presenting  %the previous author is presenting the paper (name becomes underlined)
\address{SINTEF Materials and Chemistry, 7465 Trondheim, NORWAY}%affiliation of the previous author
\email{ola.nordmann@sintef.no}%e-mail address of the previous author
\author{Zhi}{L. Xie}
\address{NTNU Department of Physics, 7491 Trondheim, NORWAY}
\email{lin.xie@ntnu.no}
%\author{Sverre}{G. Johnsen}
%\address{SINTEF Materials and Chemistry, 7465 Trondheim, NORWAY}
%\email{sverre.g.johnsen@sintef.no}

\newcommand{\TODO}[1]{\textcolor{blue}{TODO: #1} \\}
\newcommand{\Fede}[1]{\textcolor{green}{#1} \\}
\newcommand{\Pablo}[1]{\textcolor{blue}{#1}}


\begin{document}
\maketitle  %create the title page
\headers   %create the page headers and footers

\newpage

\section{Introduction}
%\Fede{FEDE WAS HERE!!! Faltan algunas referencias, y tenemos que coordinarlas. Ponele, mi primer paper en CFD lo cito como Sporleder2010, pero creo que vos lo ten\'es como "sporleder" Intent\'e abrir el .bib con mi JabRef, pero me tir\'o error. Propongo indicar cambios que cada uno haga en las secciones en com\'un o secciones ajenas (si yo agrego una ecuaci\'on al explicar LSQ no indico nada, pero si lo hago en la intro s\'i, x ej.). Ah, y me sigue tirando errores de paquetes. A vos no? Tengo 13 errores y 33 bad boxes ahora :-D }

Theories for studying porous media were first formulated as early the end of the eighteen century after the volume fraction concept was introduced for describing mud. Half a century later one of the most basic law for saturated porous solids was established by Henry Darcy, i.e. the flow velocity of the liquid in a porous solid is proportional to the pressure gradient. Since then more complex theories for predicting the proportionality constant, incorporating the capillary force and considering deformable media with the definition of the consolidation problem \citep{Boer1992}. 

Studies on the water-soil characteristic curve \citep{durner1994}, experimental characterization techniques such as gas adsorption or mercury porosimetry \citep{Navas,Jaroniec1997,Abell1999}(nitrogen adsorption and mercury) and the popularization of pore networks \citep{zhang1994, held2001, blunt2002, piri2005, ahrenholz2008, sholokhova2009} have solidify the concept of coexisting different-size pores, better represented as the pore size distribution, e.g. see \cite{dullien1991}.

Indeed, there are many different phenomena that will be affected by the pore size, as for example the physical principles of any of the experimental methods mentioned before. At the same time, heterogeneous reactions and the pore hydraulic resistivity are expected to depend strongly with the pore size. Therefore there is a clear need for adopting models that are able to capture and use the information on the pore size distribution in an efficient way.

The multi-scale transport equation for porous media (MSTE) was proposed as an extension of approaches considering a finite number of groups as studied for example by \cite{Chen1989}, \cite{bouffard2001}. \cite{DupuySchwarz} introduced one internal coordinate to the continuity problem for accounting the multi scale transport of species and the scale-dependent physics. Furthermore, the MSTE defines a inter-pore redistribution function $h$ that makes the equation open and requires further research to find the most adequate closure or model.

However, the incorporation of one internal coordinate to the transport equation introduces one extra dimension in complexity. Not only all the main variables, depend now on the pore scale, but the partial differential equation system is now an integro-partial differential equation system. This brings problems at the time of choosing the most suitable solution method. 

Many other phenomena in nature can be described by an integro-differential equation with one or many internal coordinates. The Boltzmann transport equation has the molecule velocity as an internal coordinate. The transport equation applied to neutrons or its condensation into the diffusion equation for neutrons use the neutron energy as internal energy. Population balance problems use the particle, bubble or droplet size as internal coordinate. The least square method has been applied succesfully to population balance problems by \cite{Dorao05a}.


\TODO{     * Introduce LSQ}


The scope of the present work does not allow us to address how to find any of the kernels of the MSTE nor any comparison with experiments. We are interested in the study of the suitability of the LSQ method for the solution of the MSTE for porous media, and this will be achieved by the manufactured solution method. We will only focus here in the steady state case, in one dimension only.

The objective of this paper is to propose and explain how to use LSQ for the solution of the multi-scale transport equation for porous media. The multi-scale transport, the solution method and a numerical example are given in the following three sections.

\section{Model}

The term pore-size have been used loosely in the literature and needs to be addressed first. \cite{Dullien} proposed to use twice the hydraulic radius as a convenient definition but others might work as well in the present approach since what we are after is a characteristic length that will represent the pore scale. Any of these choices requires a proper definition of a pore as a unity, for which any arbitrary definition of a portion of the void volume can chosen where a pore size can be measure (e.g. hydraulic radius) for each pore. Most pore networks agree on the definition of these elementary units. Each pore will therefore have a volume. In the MSTE, all the void volume is assigned to pores, and the concept of throat is not needed making a difference with some pore network strategies. Let's define $\xi$ as the pore size, and $\epsilon(\xi)$ the summed volume for all the pores of size $\xi$ per volume unit.

On average, the permeability will be different for each pore size as well, $k(\xi)$. A driven force will produce an advective velocity of the fluid at the scale $\xi$ denoted as $v(\xi)$, and the discharge (or slip velocity) for the same scale is $q(\xi)=\epsilon(\xi) v(\xi)$.

\cite{DupuySchwarz} introduce the concept of the redistribution function $h(\xi_1 \rightarrow \xi_2)$ as the discharge from pores of size $\xi_1$ into pores of sizes $\xi_2$. This is a measure of the mixing between pores of different scales.

The $\epsilon(\xi)d\xi$ and $q(\xi)d\xi$ are the probability density functions for porosity and discharge. Their definition is sometimes clarified by defining the cumulative distribution function, $F_{\epsilon}(\xi)=\int_0^{\xi}\epsilon(\hat \xi)d\hat\xi$ and $F_{q}(\xi)=\int_0^{\xi}q(\hat\xi)d\hat\xi$. $F_{\epsilon}(\xi)$ is dimensionless and represents the porosity of a solid media obtained by considering all the pores of size smaller or equal to $\xi$.

The multi-scale transport for a tracer, specie $i$, in all the pores assuming one dimension ($z$) and steady state case can be written as \citep{DupuySchwarz}:

\begin{align}
q(z, \xi) \cdot \frac{\partial \phi_{i}(z, \xi)}{\partial z}  = 
 S_{i}(z, \xi) + \int [\phi_{i}(z, \hat \xi) - \phi_{i}(z, \xi)] h(z, \hat \xi \rightarrow \xi) d\hat \xi . 
\label{eq:ss2}
\end{align}

The left hand side and first term of the right hand side of the equation \ref{eq:ss2} are the convection per scale and the source per scale and present no major interaction between scales. However, the integral on the right hand side couples different scales and makes impossible to solve the problem for each scale independently. It should be pointed out that advection can be happening at different speeds ($\tfrac{\partial q}{\partial \xi} \neq 0$) and the source term can also be written independently for each pore-size.

\TODO{    * Point out challenges for solving the equation}

\section{The least-squares method}
Let us write our system of equations as follows:

\begin{eqnarray}
\mathcal{L} \phi = \mathbf{g} \quad \mbox{in} \ \Omega \label{eq:Problem} \\
\mathcal{B} \phi = \mathbf{f_\Gamma} \quad \mbox{on} \ \Gamma \label{eq:Boundary}
\end{eqnarray}

\begin{eqnarray}
\mathcal{L} \mathbf{f} = \mathbf{g} \quad \mbox{in} \ \Omega \label{eq:Problem} \\
\mathcal{B} \mathbf{f} = \mathbf{f_\Gamma} \quad \mbox{on} \ \Gamma \label{eq:Boundary}
\end{eqnarray}

\noindent where $\Omega$ is the domain of the system, $\Gamma$ is the boundary of the domain, $\mathcal{L}$ is a functional operator, and $\mathcal{B}$ is the trace operator. Equation \ref{eq:Boundary} represents the equations of the boundary conditions, while  equation \ref{eq:Problem} represents the main equations of the problem. 

In this manner, \Pablo{a} norm equivalent functional \Pablo{can} defined as follows:

\begin{eqnarray} 
\mathcal{J}(\mathbf{f})\equiv  \frac{1}{2}\parallel \mathcal{L} \mathbf{f} -\mathbf{g} \parallel_{Y(\Omega)}^2 + \frac{1}{2} \parallel \mathcal{B}\mathbf{f} - \mathbf{f}_{\Gamma} \parallel_{Y(\Gamma)}^2. 
\label{eq:WeakLSQ}
\end{eqnarray}


\begin{eqnarray} 
\mathcal{J}(\mathbf{\bullet})\equiv  \frac{1}{2}\parallel \mathcal{L} \bullet -\mathbf{g} \parallel_{Y(\Omega)}^2 + \frac{1}{2} \parallel \mathcal{B}\bullet - \mathbf{f}_{\Gamma} \parallel_{Y(\Gamma)}^2. 
\label{eq:WeakLSQ3}
\end{eqnarray}

\Pablo{The functional $\mathcal{J}(\mathbf{f})$} is a measure of the residual of an aproximating function $\mathbf{f}\in X(\Omega)$.
The residual of the boundary equations is included in equation \ref{eq:WeakLSQ} \Pablo{(second term of the RHS)} to allow a weak imposition of the boundary conditions. That is, we allow the use of spaces $X(\Omega)$ that are not constrained to satisfy these conditions.

It is assumed that the problem is well-posed and that the operators $\mathcal{L}$ and $\mathcal{B}$ conform a continuous mapping of the function space $X(\Omega)\times X(\Gamma)$ on the space $Y(\Omega) \times Y(\Gamma)$ \Pablo{EEEEhh.... quien es Y?}. Under these conditions, a norm equivalence between the norm of the residual and the norm of the error is established. That is, finding a function $\mathbf{f}\in X(\Omega)$ that minimizes the functional $\mathcal{J}(\mathbf{f})$ is equivalent to finding the approximating function in $X(\Omega)$ which is closest to the exact solution to the problem.

\Pablo{Yo lo entiendo porque yo se LSQ (algo al menos)... pero no se como se explica mejor realmente... $\mathbf{f}$ es la solucion o es la aproximacion a la solucion? equacion 2 es inconsistente con el resto... o no? Es una muy poco feliz decision elegir el mismo symbolo para ambas, o no? Debido a que en nuestro problema la solucion es $\phi$ decidi hacer el cambio en la definicion del problema. Luego usamos $\mathbf{f}$ para la aprox solo.  Agregue un texto que capaz ayuda para aclarar ese punto.}

Therefore, the least squares method is based on finding an $\mathbf{f} \in X(\Omega)$ such that this functional is minimized. 
%
%The spectral element method of the least squares type consists in sub-dividing the computational domain $\Omega$ into $N_e$ non--overlapping sub--domains $\Omega_e$, called spectral elements. We then look for the minimizing function within each spectral element $\mathbf{f}_P^e \in X_P(\Omega_e) \subset X(\Omega_e)$ such that:
%
%\begin{eqnarray}
%\mathbf{f}_P^e&=&\sum_{j=0}^{N_p-1} \mathbf{b}_j^e \varphi_j^e \label{eq:suma} \\
%\mathbf{f}_P&=&\bigcup _{e=1}^{N_e}\mathbf{f}_P^e
%\end{eqnarray}
%
%\noindent where $\mathbf{b}_j^e$ are the coefficients of the expansion for the spectral element $e$. 

\subsection{Solution procedure}

\Pablo{No he leido esta seccion todavia en mucho detalle. Pero parece estar bien. Si es la primera vez que se publica este tipo de informacion decime que agarro la lupa y la miro en mas detalle... se la podes dar a Hugo a ver si la entiende, jeje }
Using variational analysis, minimizing the functional $\mathcal{J}(\mathbf{f})$ is equivalent to finding $\mathbf{f} \in X(\Omega)$ such that:

\begin{eqnarray}
\lim_{\epsilon \to 0} \frac{d}{d\epsilon}{\mathcal{J}(\mathbf{f}+\epsilon \ \mathbf{w})}=0 && \forall \mathbf{w} \in X(\Omega)
\label{eq:minimization}
\end{eqnarray}

\noindent where $X(\Omega)$ is the space of admissible functions.

We use the $L^2$ norm defined as $\parallel \bullet  \parallel_{Y(\Omega)}^2= \langle \bullet, \bullet \rangle= \int_\Omega  \bullet \bullet   \ d\Omega$ for simplicity. Using $\mathcal{E} {(}\mathbf{f} {)}=\mathcal{L}\mathbf{f}-\mathbf{g}$ to represent the equations, we express the norm equivalent functional as:

\begin{equation}
\mathcal{J}(\mathbf{f})\equiv  \frac{1}{2} \langle \mathcal{E} {(}\mathbf{f} {)},\mathcal{E} {(}\mathbf{f} {)} \rangle
\end{equation}

Here, we leave aside the equations for the boundary conditions. The extension of the solution procedure to these is straight-forward. 

In practice, we restrict the search of the minimizing function to a finite--dimensional function space such that $\mathbf{f}_P \in X_P(\Omega) \subset X(\Omega)$ and $X_P(\Omega)= span\{ \varphi_0,...,\varphi_P \} $:

\begin{eqnarray}
\mathbf{f}_P^e&=&\sum_{j=0}^{N_p-1} \mathbf{b}_j^e \varphi_j^e \label{eq:suma} \\
\mathbf{f}_P&=&\bigcup _{e=1}^{N_e}\mathbf{f}_P^e
\end{eqnarray}


Equation \ref{eq:minimization} yields then the equivalent problem: find $\mathbf{f_P} \in X_P(\Omega)$ such that:

\begin{equation}
\big\langle \frac{\partial\mathcal{E}}{\partial \mathbf{f}}\mathbf{w} , \mathcal{E}(\mathbf{f_P}) \big\rangle=\mathcal{G}=0 \ ,\ \ \forall \mathbf{w}\in X_P(\Omega)
\end{equation}


Here, $\frac{\partial\mathcal{E}}{\partial \mathbf{f}}$ is the operator obtained by dividing the different equations by the different variables. It can be divided in rows which represent each equation, and in columns which represent the operator affecting each variable in each equation. That is,

\begin{eqnarray}
\mathbf{f}=\left[ \begin{array}{ccccc} f_1 & ... & f_{v} & ... & f_{N_v} \end{array} \right]^T \nonumber\\
 {\mathcal{E}} {\frac{\partial\mathcal{E}}{\partial \mathbf{f}}}=\left[ \begin{array}{ccccc}
 {\frac{\partial\mathcal{E}_{1}}{\partial \mathbf{f_1}}} & ... &  {\frac{\partial\mathcal{E}_{1}}{\partial \mathbf{f_v}}} & ... &  {\frac{\partial\mathcal{E}_{1}}{\partial \mathbf{f_{N_v}}}} \\
... & ... & ... & ... & ... \\
 {\frac{\partial\mathcal{E}_{c}}{\partial \mathbf{f_{1}}}} & ... &  {\frac{\partial\mathcal{E}_{c}}{\partial \mathbf{f_{v}}}} & ... &  {\frac{\partial\mathcal{E}_{c}}{\partial \mathbf{f_{N_v}}}} \\
... & ... & ... & ... & ... \\
 {\frac{\partial\mathcal{E}_{N_c}}{\partial \mathbf{f_{1}}}} & ... &  {\frac{\partial\mathcal{E}_{N_c}}{\partial \mathbf{f_{v}}}} & ... &  {\frac{\partial\mathcal{E}_{N_c}}{\partial \mathbf{f_{N_v}}}} \\
\end{array} \right] 
\label{eq:eqOp}
\end{eqnarray}

\noindent where $N_v$ is the number of variables and $N_c$ is the number of equations.

For linear problems, $\frac{\partial\mathcal{E}}{\partial \mathbf{f}}\mathbf{w}=\mathcal{L}\mathbf{w}$, and the problem is reduced to finding $\mathbf{f_P} \in X_P(\Omega)$ such that:


\begin{eqnarray}
\mathcal{A}(\mathbf{f_P},\mathbf{w})=\mathcal{F}(\mathbf{w}) \ ,\ \forall \mathbf{w} \ \in X_P(\Omega)
\label{eq:equivalent}
\end{eqnarray}

\noindent where

\begin{eqnarray}
\mathcal{A}(\mathbf{f},\mathbf{w})= \langle \mathcal{L}\mathbf{f},\mathcal{L}\mathbf{w} \rangle \nonumber \\ 
F(\mathbf{w})= \langle \mathbf{g},\mathcal{L}\mathbf{w} \rangle 
\label{eq:equivLinear}
\end{eqnarray}




For non-linear problems, a seed solution $\mathbf{f^*_P}$ is proposed, and the Newton-Raphson method is used to iterate until the minimizing solution is found. This method is based on the expansion in Taylor series of $\mathcal{G}$ around $\mathbf{f^*_P}$, and yields the following equivalent problem: find $\delta\mathbf{f_P}=\mathbf{f_P}-\mathbf{f^*_P} \in X_P(\Omega)$ such that:

\begin{eqnarray}
\big\langle \frac{\partial}{\partial \mathbf{f}} \left(\frac{\partial\mathcal{E}}{\partial \mathbf{f}}\mathbf{w}\right)^*\mathbf{\delta f_P} , \mathcal{E}(\mathbf{f^*_P}) \big\rangle  \nonumber \\
+\big\langle \frac{\partial \mathcal{E}}{\partial \mathbf{f}}^*\mathbf{\delta f_P} , \frac{\partial \mathcal{E}}{\partial \mathbf{f}}^*\mathbf{w} \big\rangle +
\big\langle \mathcal{E}(\mathbf{f^*_P}) , \frac{\partial \mathcal{E}}{\partial \mathbf{f}}^*\mathbf{w} \big\rangle =0 \ ,\ \mathbf{w} \ \in X_P(\Omega)
\end{eqnarray}

The first term is usually set to zero \cite{Winterscheidt1994, Codd2001}. This term is dominated by the others near the solution. Furthermore, the absence of this term simplifies greatly the computation and does not affect the method, only the way in which the solution is looked for.

The equivalent problem to be solved is to find $\mathbf{f_P} \in X_P(\Omega)$ such that equation \ref{eq:equivalent} is fulfilled, with:


\begin{eqnarray}
\mathcal{A}(\mathbf{f},\mathbf{w})= \big\langle \frac{\partial \mathcal{E}}{\partial \mathbf{f}}^*\mathbf{f} ,  \frac{\partial \mathcal{E}}{\partial \mathbf{f}}^*\mathbf{w} \big\rangle \nonumber \\
F(\mathbf{w})= \big\langle \frac{\partial \mathcal{E}}{\partial \mathbf{f}}^*\mathbf{f^*}-\mathcal{E}(\mathbf{f^*}) , \frac{\partial \mathcal{E}}{\partial \mathbf{f}}^*\mathbf{w} \big\rangle 
\label{eq:equivNR}
\end{eqnarray}

%\begin{equation}
%\big\langle \frac{\partial \mathcal{E}}{\partial \mathbf{f}}|_0\mathbf{f} ,  \frac{\partial \mathcal{E}}{\partial \mathbf{f}}|_0\mathbf{w} \big\rangle =
%\big\langle \frac{\partial \mathcal{E}}{\partial \mathbf{f}}|_0\mathbf{f_0}-\mathcal{E}\mathbf{f_0} , \frac{\partial \mathcal{E}}{\partial \mathbf{f}}|_0\mathbf{w} \big\rangle\ \ , \ \forall \mathbf{w} \in X(\Omega)
%\end{equation}

Comparing equation \ref{eq:equivNR} to equation \ref{eq:equivLinear}, this procedure is equivalent to linearizing the equations using the Newton-Raphson method and then applying the least-squares method to the resulting linear problem $ \frac{\partial \mathcal{E}}{\partial \mathbf{f}}^*\mathbf{f}=\frac{\partial \mathcal{E}}{\partial \mathbf{f}}^*\mathbf{f^*}-\mathcal{E}(\mathbf{f^*})$. The Picard method (also known as fixed point or direct substitution method) has also been used to linearize the equations \cite{Sporleder2010}.

The algebraic set of equations corresponding to the equivalent problem is found expanding the approximate solution (equation \ref{eq:suma}). The internal products are approximated numerically using quadrature rules. Thus, the system of algebraic equations is given as follows:

\begin{equation}
\mathbf{A} \mathbf{b} = \mathbf{F}
\end{equation}

\noindent where $\mathbf{b}$ is the vector of coefficients of the expansion in equation \ref{eq:suma}, and where:

\begin{eqnarray}
\mathbf{A}=\mathbf{L}^T\mathbf{W}\mathbf{L} \nonumber \\
\mathbf{F}=\mathbf{L}^T\mathbf{W}\mathbf{G} \nonumber
\end{eqnarray}

$\mathbf{L}$ is called the problem operator matrix, $\mathbf{W}$ is a diagonal matrix built with the quadrature weights and $\mathbf{G}$ is called source vector. The three can be subdivided analogously to equation \ref{eq:eqOp}:

\begin{eqnarray}
\mathbf{G}=\left[ \begin{array}{ccccc} G_1 & ... & G_{c} & ... & G_{N_c} \end{array} \right]^T \nonumber\\
\mathbf{L}=\left[ \begin{array}{ccccc}
L_{11} & ... & L_{1v} & ... & L_{1N_v} \\
... & ... & ... & ... & ... \\
L_{c1} & ... & L_{cv} & ... & L_{cN_v} \\
... & ... & ... & ... & ... \\
L_{N_c1} & ... & L_{N_cv} & ... & L_{N_cN_v} \\
\end{array} \right] 
\end{eqnarray}

\noindent where:

\begin{eqnarray}
\left[L_{cv}\right]_{qk}&=&\left(\frac{\partial \mathcal{E}_c}{\partial \mathbf{f}}\right)^*_v \varphi_k (\mathbf{r_q}) \nonumber \\
\left[G_{c}\right]_{q}&=&\frac{\partial \mathcal{E}_c}{\partial \mathbf{f}}^*\mathbf{f_P^*}(\mathbf{r_q})-\mathcal{E}_c(\mathbf{f_P^*}(\mathbf{r_q}))
\end{eqnarray}

Here, $\mathbf{r_q}$ is the quadrature point $q$.

\section{Numerical examples}


We select the case of flow through a column for our numerical example. A one metre column filled with a packed bed of rocks has a porosity per scale given by $\epsilon({\xi})$. Liquid is introduced from the top which percoles down the column at a given discharge per scale given by $q({\xi})$. The pore-scale redistribution, $h$  is assumed proportional to the discharges.

In the manufacturing solution method we propose a solution $\phi(\xi, z)$ and calculate analitically the value of the source. For this problem the chosen values are:

\begin{align}
\epsilon(\xi)=&A (\xi-\xi_0) (-\xi+\xi_f) \\
q({\xi}) =&V \epsilon(\xi) \xi^2\\
h(\hat \xi, \xi, z) =& q(\hat \xi) q(\xi)/L\\
\phi(\xi, z) =& 10 z (1-M Log({\xi}/R))\\
\nonumber S(\xi, z)=& 10 A V {\xi}^2 ({\xi}-\xi_0) (-{\xi}+\xi_f) \left(1-M \text{Log}\left[\frac{{\xi}}{R}\right]\right) +\\
\nonumber  &A^2 M V^2  (360 L)^{-1} {\xi}^2 ({\xi}-\xi_0) ({\xi}-\xi_f) z \times ... \\
\nonumber  &\big(-81 \xi_0^5+175 \xi_0^4 \xi_f-175 \xi_0 \xi_f^4+  ...\\
\nonumber &81 \xi_f^5+60 \xi_0^4 (3 \xi_0-5 \xi_f) \text{Log}\left[\frac{\xi_0}{{\xi}}\right]+ ...\\
& 60 (5 \xi_0-3 \xi_f) \xi_f^4 \text{Log}\left[\frac{\xi_f}{{\xi}}\right]\big)
\end{align}

We rewrite the problem equation \ref{eq:ss2} as,

\begin{eqnarray}
q(z,\psi)\frac{\partial \phi(z,\xi)}{dz} +\phi(z,\xi)\int h(z,\xi\rightarrow\hat{\xi})d\hat{\xi} \nonumber \\
-\int \phi(z,\hat{\xi}) h(z,\xi\rightarrow\hat{\xi})d\hat{\xi} = S(z,\xi)
\end{eqnarray}

The equation is linear when solved for $\phi(z,\xi)$. Thus, $\frac{\partial \mathcal{E}}{\partial \phi}\bullet=\mathcal{L}\bullet=q(z,\psi)\frac{\partial \bullet}{dz} +\int h(z,\xi\rightarrow\hat{\xi})d\hat{\xi}\bullet -\int h(z,\xi\rightarrow\hat{\xi})\bullet d\hat{\xi}$. The equation presents a derivative, an identity and an integral operator applied to $\phi$. In matricial form:

\begin{eqnarray}
\mathbf{L}&=&\mathbf{C_1}\mathbf{Dz}+\mathbf{C_2}\mathbf{H} \Pablo{+} \mathbf{I_h} \\
\mathbf{G}&=&\mathbf{S}
\end{eqnarray}

\noindent with:

\begin{eqnarray}
\left[\mathbf{C_1}\right]_{qj}&=& \delta_{qj}q(z_q,\xi_q) \nonumber \\
\left[\mathbf{C_2}\right]_{qj}&=& \delta_{qj}\int h(z_q,\xi_q\rightarrow\hat{\xi})d\hat{\xi} \nonumber \\
\left[\mathbf{S}\right]_{q}&=& S(z_q,\xi_q) \nonumber 
\end{eqnarray}

\noindent where $\delta_{qj}$ is the Kronecker delta. The other matrices, namely $\mathbf{D_z}$, $\mathbf{H}$, and $\mathbf{I_h}$ are called operator matrices.


$\mathbf{D_z}$ is called the derivative operator matrix in z, and is defined as follows:

\begin{equation}
\left[\mathbf{D_z}\right]_{qj}=\frac{\partial \varphi_j}{\partial z}(z_q,\xi_q)
\end{equation}

$\mathbf{H}$ is the identity operator matrix, and is defined as follows:

\begin{equation}
\left[\mathbf{H}\right]_{qj}=\varphi_j(z_q,\xi_q)
\end{equation}

$\mathbf{I_h}$ is the integral operator weighted with $h$, and is defined as follows:

\begin{eqnarray}
\left[\mathbf{I_h}\right]_{qj}&=& \Pablo{-} \int h(z_q,\xi_q\rightarrow\hat{\xi})\varphi_j(z_q,\hat{\xi}) d\hat{\xi}\nonumber \\
&=& \Pablo{-} \sum_k h(\xi_q\rightarrow s_k \rightarrow\xi_q)\varphi_j(z_q,s_k) \omega_k
\end{eqnarray}

\Pablo{No hay que borrar el segundo $\xi$ dentro de h, del segundo renglon?}

Here, we define new quadrature points and weights such that $s_k$ exists within the integration boundaries for each row of the integral operator matrix. For \Pablo{the present problem the integration limits are the domain limits.}

\section{Results}


\TODO{Mention implementation details (polynomials and quadratures used, number of points, solver, etc.).  ouch... GLL, 12 puntos como se ve en la figura} \Pablo{Ya esta, podemos quitar este TODO?}

\noindent Choosing the values $A=\tfrac{1}{500}$, $V=0.001$, $M=1$ and pore scale limits $\xi_0=1$ and $\xi_f=10$ mm. Gauss-Legendre-Lobato points where chosen for the external coordinate and Gauss-Legendre for the internal coordinate. \Pablo{revisa que tenga sentido esta oracion por favor...}

%\TODO{Explain and show q, epsilon and h. and BCs}

Both the discharge, $q({\xi})$ and the porosity, $\epsilon({\xi})$ do not depend on the position along the column. Plot of the distribution and cumulative distribution for these two functions are given in Figure \ref{fig:profiles}.

\InsFig{profiles}{Left column: porosity distribution (top) and cumulative porosity distribution (bottom). Middle column: discharge distribution (top) and cumulative discharge distribution (bottom). Right column: velocity per channel (top), and 1-metre residence time (bottom)}{profiles}

%\TODO{Describe what case could be described with this simplified model.}

The solution is shown in Figure \ref{fig:solution}. As the liquid flows down the column $\phi$ increases due to reactions happening inside the pores. Despite the source term is very different for different pore sizes (negative for large pores and positive for small pores, see Figure \ref{fig:source}) the tracer $\phi$ is exchanged between different pores scales tending to level out the differences. However the increase is larger for the small pores at all the length of the column (z values). The source term (Figure \ref{fig:source}) behaves differently for different pore scales. Large amounts of $\phi$ are generated for pores smaller than 5 mm, while a sink is observed in the largest pores. This behaviour for the source term mimic two competing reactions. It models the case of a specie that is a product of an heterogeneous reaction but is a reactant of a reaction taking place in the liquid bulk. Heterogeneous reactions depend strongly of the surface to volume ratio which is much larger for small pore scales. At the same time the reactant is consumed in the larger pores in this case as the bulk reaction that consumes $\phi$ becomes more dominant.

\InsFig{solution}{Solution $\phi(z,\xi)$.}{solution}

\InsFig{source}{Source term $S(z,\xi)$. }{source}

%\TODO{Explain what is happening with phi and S in each scale, and the mixing by h.}

%\TODO{Show and comment on convergence plot}
Finally, the polynomial order for both dimensions was varied from 5 to 12 to verify if the numerical solution approximates to the proposed analytical solution. The infinity norm error is shown in the Figure \ref{fig:convergence} where it can be seen that the error is reduced exponentially with the method order up to computer precision.


\InsFig{convergence}{Convergence plot}{convergence}

\section{Conclusion}

The conclusions of this paper are:

\begin{itemize}
\item The least square high order method is well suited for the solution of the multi-scale transport equation for porous media achieving exponential convergence in the method order.
\item A method has been described and can now be applied for the solution of transport and reactions problems in porous media.
\end{itemize}

Future work still remains on the area of finding the most adequate kernels that best represent different porous media. Mainly empirical studies of the redistribution function need to be done.

%\abstract{
%  This file is an example \LaTeX file for submission to CFD2011.  A
%  limit of 10 pages applies.
%}
%\keywords{
%  CFD, hydrodynamics, chemical reactors
%}
%\normalfont\normalsize
%
%
%
%%\section{Nomenclature}
%A complete list of symbols used, with dimensions, is required.
%%the nomenclature needs to be entered into the file ExampleFile.nls
%\printnomenclature[0.7cm]  
%\vskip .1em
%
%
%\section{Introduction}
%The introduction goes here.
%
%
%\newpage
%
%
%
%\section{Model Description}
%You should give a thorough description of your model.
%\vspace{4cm}
%\subsection{Example of Subheading}
%Here is how to produce a numbered equation under a second level
%heading \cite{James1988}.
%\vspace{2cm}\\
%\emph{Continuity equation}
%\begin{equation}
%  \frac{\partial \rho_G}{\partial t}+\nabla\left(\rho_G\mathbf
%    u\right)=0
%\end{equation}
%
%\subsubsection*{Example of Sub-subheading}
%This is how \cite{Luke1988} produced an unnumbered equation under a
%third level heading.
%\begin{equation}
%  \mathbf J=\sigma(\mathbf E+\mathbf u\times\mathbf B)
%\end{equation}
%\newpage
%

%\InsFig{figure}{Schematic diagram of geometry.}{label1}
%\InsFigRot{figure}{Rotated schematic diagram of geometry.}{label2}
%
%
%\section{Results}
%The results of using the \LaTeX template is a great looking paper.
%In Figures \ref{fig:label1} and \ref{fig:label2} it can be seen how figures are easily included.
%In Table \ref{tab:label} it is seen how we can include a table.
%The table is constructed in the file table.tex, where also the table caption and label are defined.
%
%
%\newpage
%\InsTab{Table}
%
%
%\section{Conclusion}
%The conclusions are:
%\begin{enumerate}
%  \item Trondheim is a nice city.
%  \item CFD is great fun, and useful too.
%\end{enumerate}
%
\newpage
%
%
%% %---------------------------
%% %BIBLIOGRAPHY
%% %---------------------------
%The bibliography is created using BiBTeX
\bibliographystyle{CFD2011}
\bibliography{MSTE}


%\newpage
%\section{Appendix A}
%List of animation files:
%s10.avi Motion of the spherical particles.
%v10.mped Flow field around each particle.


\end{document}